\documentclass{beamer}
\usepackage[utf8]{inputenc}
\usepackage{tikz}
\usepackage{graphicx}
\usetheme{Warsaw}

\title{Semaine du Numérique}
\author{Vincent Berry, Grégoire Guisez, Jamal Hennani, Victor Hiairrassary, William Tassoux}
\institute{Polytech' Montpellier, Lirmm}
\date{2 mai 2012}

% On enlève la barre de navigation
\setbeamertemplate{navigation symbols}{}

% TODO: Afficher le plan!


\begin{document}

% On diminue la police
\small 


\begin{frame}
    \titlepage
\end{frame}



\begin{frame}{Introduction}
        \begin{block}{Introduction}
 		 Ce projet vise à mettre en place une partie de la semaine du numérique qui se déroulera lors des années à venir.
		 On peut le décomposer en 2 sous projets:
		\begin{itemize}
			\item Le badger
			\item Le site web
		\end{itemize}
	\end{block}

    	\begin{block}{SDM}
 		 La semaine du numérique correspond à un évenement ponctuel coordonnée par polytech, en collaboration avec des intervenants extérieurs. 			 Cette derniere sera sanctionné comme tout autre unité d'eseignement par un QCM.
	\end{block}	
\end{frame}


\begin{frame}{plan}

 	\tableofcontents
\end{frame}

\begin{frame}{developpement}
 \section{developpement}

	\begin{block}{github}
		\begin{itemize}
			\item git: Logiciel de gestion de version
			\item github: site web collaboratif
		\end{itemize}
	\end{block}

	\begin{block}{wamp xampp}
		\begin{itemize}
			\item serveur web en localhost
			\item phpMyAdmin
			\item Apache
			\item MySQL
		\end{itemize}
	\end{block}

	\begin{block}{c++}
		\begin{itemize}
			\item code blocks
			\item Xcode
		\end{itemize}
	\end{block}

\end{frame}

\begin{frame}{badger}
 \section{badger}
  \subsection{fonctionnalité}
	\begin{block}{fonctionnalité}
		\begin{itemize}
			\item controle présence éleve
			\item atout: rapidité et éfficacité
		\end{itemize}
	\end{block}

 \subsection{fonctionnement}
	\begin{block}{fonctionnement}
		\begin{itemize}
			\item serie (rs232)
			\item échange données PC-badger: bluethoot
			\item echange données badger BDD polytech : internet (MySql)
			\item atout: multiplateforme(windows.h,termios.h,???)
		\end{itemize}
	\end{block}

\end{frame}

\begin{frame}{site web}
 \section{site web}
  \subsection{utilité}
	\begin{block}{utilité}
		Etudiant:
		\begin{itemize}
			\item inscriptions aux conférences
			\item visualisation de cours en ligne
			\item passage QCM
		\end{itemize}
		Administrateur:		
		\begin{itemize}
			\item gestion des inscrits
			\item gestion des salles
			\item upload de cours
			\item gestion des notes
		\end{itemize}
	\end{block}
			
\end{frame}

\begin{frame}{architecture MVC}
 \subsection{architecture MVC}
	L'architecture MVC correspond à une maniere d'organiser le code afin de séparer:
	\begin{itemize}
		\item les données
		\item leurs traitements
		\item la présentation
	\end{itemize}

	Ce qui correspondra à :
	\begin{itemize}
		\item modèle
		\item vue
		\item contrôleur
	\end{itemize}
\end{frame}

\begin{frame}{modèle}
   \subsubsection{modèle}
	\begin{block}{modèle}
	Gestion de l'organisation des données, c'est à dire traitements des données:
		\begin{itemize}
			\item requêtes SQL
			\item fichier xml
		\end{itemize}
	\end{block}
\end{frame}


\begin{frame}{vue}
   \subsubsection{vue}
	\begin{block}{vue}
	Correspond à l'interface avec laquelle intéragit l'utilisateur
		\begin{itemize}
			\item interface homme/machine
			\item contient le code html
			\item fichier xml
		\end{itemize}
	\end{block}	
\end{frame}


\begin{frame}{contrôleur}
   \subsubsection{contrôleur}
	\begin{block}{contrôleur}
	Permet la mise en lien du modèle et du controleur:
		\begin{itemize}
			\item met à jour la vue et/ou le modèle
			\item répond aux évenements de l'utilisateur
		\end{itemize}
	\end{block}
			
\end{frame}


\begin{frame}{interface administrateur}
  \subsection {interface administrateur}
	\begin{block}
		aa
	\end{block}
\end{frame}


\begin{frame}{interface utilisateur}
  \subsection {interface utilisateur}
	\begin{block}
		aa
	\end{block}
\end{frame}

\begin{frame}{visionneuse}
  \subsubsection {visionneuse}
	\begin{block}
		aa
	\end{block}
\end{frame}

\end{document}
