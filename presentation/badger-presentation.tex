\documentclass[handout]{beamer}

\usepackage[utf8]{inputenc}
\usepackage[frenchb]{babel}
\usepackage{amsmath}
\usepackage{graphicx}
\usepackage{tikz}

\title{Semaine du Numérique}
\author[G. Guisez, J. Hennani, V. Hiairrassary, W. Tassoux]{Grégoire Guisez, Jamal Hennani, Victor Hiairrassary, William Tassoux}
\institute{Polytech' Montpellier, Université Montpellier 2}
\date{15 mai 2012}

\usetheme{Warsaw}

\AtBeginSection[]{
   \begin{frame}
   \begin{center}{\Large Plan }\end{center}
       \tableofcontents[currentsection,hideothersubsections]
   \end{frame} 
}


\begin{document}

\small 

%\logo{\includegraphics[scale=0.2]{images/logoPolytech.jpg}}
%\setbeamertemplate{sidebar left}
%{
%    \logo{\includegraphics[scale=0.05]{images/logoUm2.jpg}}
%    \vfill
%    \rlap{\hskip0.1cm\insertlogo}
%    \vskip8pt
%}


\begin{frame}
    \titlepage

    \includegraphics[scale=0.2]{images/logoPolytech.jpg}
    \hfill
    \hskip8pt
    \includegraphics[scale=0.05]{images/logoUm2.jpg}
\end{frame}


\begin{frame}{Introduction}
    \begin{block}{Introduction}
 		 Ce projet vise à mettre en place une partie de la semaine du numérique qui se déroulera lors des années à venir.

		 On peut le décomposer en deux sous projets:
		\begin{itemize}
			\item Lecteur de cartes
			\item Site web
		\end{itemize}
	\end{block}

	\begin{block}{Semaine du Numérique}
    La semaine du numérique correspond à un évenement ponctuel coordonné
    par Polytech, en collaboration avec des intervenants extérieurs.
    Cette unité d'enseignement sera sanctionnée par un QCM.
	\end{block}	
\end{frame}


\begin{frame}{Plan}
 	\tableofcontents
\end{frame}

\chapter{Site web}

    \section{Timeline de la Semaine du Numérique}

Afin de présenter au mieux les objectifs du site Web, il est important de comprendre
comment est censé se dérouler la Semaine du Numérique.

Les packages (appelés aussi cycles de conférences) correspondent à un ensemble de
conférences regroupées autour d'un même thème.
Chaque étudiant peut donc choisir parmi le panel disponible, c'est donc un système
"à la carte", permettant ainsi de personnaliser cette unité d'enseignement.

Les étudiants doivent s'inscrire via la plateforme web à des packages avant la date
limite, une fois cette date dépassée, il ne leur est plus possible de modifier leur choix.

L'administrateur devra si nécessaire régler les problèmes d'inscriptions pour certains
étudiants avant que les conférences ne commencent.

La Semaine du Numérique consiste donc en une semaine banalisée pour les différentes
sections de Polytech' Montpellier où des intervenants aborderont certains thèmes liés
à l'informatique au coeur de l'entreprise.

Une fois les conférences passées, les étudiants doivent passer un QCM généré en
fonction des packages qu'ils ont choisi. Ils devront également rendre un rapport
pour chaque package auxquels ils se sont inscrits.

    \section{Cahier des charges}

Le site Web que nous avons développé est une plateforme de communication entre les
étudiants et Mr Berry, en charge de la Semaine du Numérique (ainsi que de notre projet).
L'application devait donc fournir un certain nombre de fonctionnalités vis à vis
des étudiants, mais aussi pour l'administration, afin de mettre en ligne les
informations et de gérer les potentiels conflits.

On notera également que la partie utilisateur devait permettre d'afficher toutes
les informations en plusieurs langues. Le framework utilisé facilite en grande partie
la réalisation d'un site multi-lingue, même si l'implémentation d'une nouvelle
langue demanderait une quantité de travail assez conséquente. Il est donc possible
de visiter intégralement le site en Français et en Anglais.

        \subsection{Partie utilisateur}

Du côté utilisateur, (également appelé Frontend), les étudiants ont accès aux sections
suivantes:

    \begin{itemize}
    \item Conférences
    \item QCM
    \item Rapports
    \end{itemize}

            \subsubsection{Conférences}

Les packages disposent d'une courte description présentant le thème, ainsi qu'une
liste de conférences. Ces conférences ont également une description, et le nom du
conférencier (et/ou la société) est également disponible.
On dispose bien sûr de la date, l'horaire et la salle où celle-ci aura lieu.

De plus, les documents en rapport avec le package peuvent soit être téléchargé
depuis le site Web, soit être consulté par la visionneuse en ligne. La création
d'une visionneuse en ligne se justifie par le fait qu'il est possible que les
conférenciers ne souhaitent pas fournir leur présentation en téléchargement direct.

Les étudiants ont donc accès à la liste des packages, et peuvent s'inscrire et
se désinscrire d'un package à tout moment dans la limite des inscriptions, mais
également dans la limite des places disponibles.
Chaque package dispose donc d'un quota de place, ainsi, le premier à faire ses choix
est le premier servi.

Afin de faciliter la navigation, une fonctionnalité permet de lister directement
les packages auxquels l'utilisateur s'est inscrit.

Les étudiants peuvent voir le planning des conférences auxquels ils sont inscrits,
encore une fois pour faciliter l'accès aux informations "pertinentes".

            \subsubsection{QCM}

Le QCM est la première manière de noter les étudiants, et surtout, la principale
raison de la création de cette interface Web. En effet, en plus d'automatiser les
inscriptions, l'utilisation d'un site Web permet de générer facilement un QCM
personnalisé pour chaque étudiant, reposant sur les conférences auxquels il s'est inscrit.

C'est l'occasion de préciser qu'une fois que l'étudiant s'est inscrit à un package,
il s'engage à être présent à chaque conférence qui le compose. En effet, les
questions sont liées à un package et non à une conférence.
Il est tout à fait possible que si un étudiant n'assiste pas à une conférence,
il ait tout de même à répondre à une question en rapport avec cette conférence.

Concernant la page de passage du QCM, il s'agit d'un formulaire classique avec des checkboxes pour chaque question.
Le bouton "Envoyer mes réponses" est protégé par une fenêtre de confirmation, afin de pouvoir revérifier ses réponses.

La note du QCM ne repose en réalité pas entièrement sur les questions posées.
Il est possible de déterminer une note de présence telle que si l'étudiant assiste
bien à chaque conférence où il est inscrit, il aura déjà une base de point pour le QCM
\newpage

La notation est basée sur le système suivant:
\begin{equation}N_{f} = N_{p} + N_{QCM}\end{equation}

\begin{equation}
    N_{p} = \sum_{i = 0}^{Nb_{inscrs.}} X_{i} \times \frac{N_{pmax}}{Nb_{inscrs.}} {,}
    \left\{
    \begin{array}{l l}
        X_{i} = 1 \text{ si }Status_{i} = \text{Present}\\
        X_{i} = -1\text{ si }Status_{i} = \text{Absent}\\
    \end{array}
    \right.
\end{equation}

\begin{equation}
    N_{QCM} = \sum_{i = 0}^{Nb_{repsVals.}} X_{i} \times RepsQuest_{i} {,}
    \left\{
    \begin{array}{l l}
        X_{i} = \frac{1}{RepsQuestV_{i}} \text{ si }Reponse_{i} = \text{Vrai}\\
        X_{i} = -\frac{1}{RepsQuestF_{i}} \text{ si }Reponse_{i} = \text{Faux}\\
    \end{array}
    \right.
\end{equation}

$$
\text{Avec respectivement:}
\left\{
\begin{array}{l l l l l l l l}
N_{f} = \text{Note finale}\\
N_{p} = \text{Note de présence}\\
N_{pmax} = \text{Note de présence maximale}\\
N_{QCM} = \text{Note de QCM}\\
Nb_{inscrs.} = \text{Nombre d'inscriptions de l'élève}\\
Status_{i} = \text{Status de la ième inscription}\\
Nb_{repsVals.} = \text{Nombres de réponses validées au QCM}\\
RepsQuest_{i} = \text{Nombres de réponses dans la question associée}\\
RepsQuestV_{i} = \text{Nombres de réponses vrai dans question la associée}\\
RepsQuestF_{i} = \text{Nombres de réponses fausse dans question la associée}\\
\end{array}
\right.
$$

    \begin{figure}[h]
        \begin{center}
        \includegraphics[scale=0.4]{images/screenshotQCM.png} 
        \end{center}
        \caption{Passage d'un QCM}
        \label{Passage d'un QCM}
    \end{figure}

Les questions qui lui sont posées ont été enregistrées. Les réponses de l'utilisateur sont également sauvées dans la base de données,
pour faciliter la décision dans le cas d'une harmonisation.

Le QCM ne peut être passé qu'une fois.

            \subsubsection{Rapports}

Les étudiants ont également pour travail de rendre un rapport faisant une synthèse des informations données lors des conférences.
Il est donc possible d'uploader pour chaque package un rapport.

        \subsection{Partie administrateur}

Il faut avoir une gestion aussi fine que possible pour pouvoir mettre en place toutes les composantes de l'application.
On dispose donc d'un menu avec ces différentes options:

    \begin{itemize}
    \item Conférences
    \item Documents
    \item QCM et inscriptions
    \item Salles
    \item Configuration
    \item Remise à zéro
    \end{itemize}

            \subsubsection{Conférences}

Pour les conférences, on upload dans un premier temps les packages de conférences via un fichier CSV (Coma Separated Values)
dans lequel les informations sont contenues de la manière suivante:

    \begin{itemize}
    \item Nombre max d'inscrits
    \item Nom français
    \item Nom anglais
    \item Description française
    \item Description anglais
    \end{itemize}

On peut ensuite pour chaque package uploader un fichier CSV contenant des conférences:

    \begin{itemize}
    \item Conférencier (et/ou entreprise)
    \item Nom français
    \item Nom anglais
    \item Description française
    \item Description anglais
    \item Date
    \item Horaire de début
    \item Horaire de fin
    \end{itemize}

Ainsi que des questions, et leurs réponses associées; ici, le CSV a un format plus spécifique, fonctionnant sur cette construction:
Une ligne contenant la question, dont la composition est la suivante:

    \begin{itemize}
    \item Question en français
    \item Question en anglais
    \item Statut (Obligatoire - Possible - Impossible)
    \end{itemize}

Chaque ligne suivante peut soit être une chaîne de caractères terminale ("\_\_vbmifare*"), ou une réponse:

    \begin{itemize}
    \item Réponse en français
    \item Réponse en anglais
    \item Bonne réponse ('T' - 'F')
    \end{itemize}

Le fichier est lu séquentiellement la question puis ses réponses et itére ainsi jusqu'à la fin du fichier.

Après avoir uploadé ces informations, il faut également pouvoir éditer en direct sans repasser par le fichier CSV.
Il y a donc un module d'édition permettant de modifier toutes les informations des packages et des conférences.
Notons que ces pages d'édition permettent donc de modifier et de supprimer les informations, et qu'il existe des pages
similaire pour toutes autres données à fournir.

Pour toutes informations de type Date et Horaires, il faut respecter les deux formats suivants:

    \begin{itemize}
    \item Date : JJ-MM-AAAA
    \item Horaire : HH:MM:SS
    \end{itemize}

Voilà une page classique d'édition, permettant une gestion fine des packages:

    \begin{figure}[h]
        \begin{center}
        \includegraphics[scale=0.4]{images/screenshotPackages.png} 
        \end{center}
        \caption{Vue des packages}
        \label{Vue des packages}
    \end{figure}

Il suffit de cliquer sur le texte afin de le rendre modifiable, et ensuite de retirer
le focus à la zone de texte, la modification est donc prise en compte.

            \subsubsection{Documents}

Cette section sert pour l'upload de documents, qu'il s'agisse de présentations directement téléchargeables, ou d'images pour la visionneuse.

L'upload pour le téléchargement est basique, il suffit d'indiquer à quel package le document réfère, et le fichier en lui même.
Dans le cas de l'upload d'images, il suffit de les regrouper dans un fichier zippé (format .zip) en faisant en sorte qu'elle soit
nommé de manière croissante. Le plus simple est probablement d'utiliser des noms de ce type:

    \begin{figure}[h]
        \begin{center}
        \includegraphics[scale=0.4]{images/screenshotImages.png} 
        \end{center}
        \caption{Visionneuse en ligne}
        \label{Visionneuse en ligne}
    \end{figure}

            \subsubsection{QCM et inscriptions}

Une fois la Semaine du Numérique finie, l'administrateur doit inscrire les promotions ayant suivi les conférences pour leur
séance de QCM Il faut sélectionner la promotion, et indiquer la date et l'horaire de la séance. Il sera alors possible aux étudiants d'accéder à la page du QCM que durant ce créneau.

Un menu d'édition sur la même base que les autres permet l'édition en direct des inscriptions.

Une fonctionnalité permet également de récupérer les notes au format CSV également, les informations étant écrites de la manière suivante:

    \begin{itemize}
    \item Département
    \item Année
    \item Nom d'utilisateur
    \item Note
    \item Commentaire
    \end{itemize}

            \subsubsection{Salles}

Il faut également indiquer les salles et les disponibilités de celles-ci pour chaque conférence, afin de pouvoir les associer.
Une interface permet directement la création, l'édition et la suppression des salles et de ses disponibilités.

            \subsubsection{Configuration}

La page de configuration permet de modifier les administrateurs du site. Ceux-ci sont enregistrés comme tels dans la base de données.
Pour ajouter un administrateur, il faut utiliser son nom d'utilisateur, et les séparer par des points-virgules.

Comme sécurité, l'administrateur qui applique cette modification est automatiquement ajouté à la liste des admins.

Il est également possible de verrouiller les inscriptions aux conférences, dans le cas où d'importantes modifications
auraient a être faites.

            \subsubsection{Remise à zéro}
La remise a zéro permet elle de vider les tables de la base de données. La suppression est \emph{définitive}, il n'y a pas de
backup possible une fois vidée. Le bouton demande une confirmation avant de réaliser cette action.

    \section{Modélisation}
Peu de temps après le début du projet, il nous a été demandé de fournir une modélisatiton du site web. De cette manière
nous pouvons affirmer ou infirmer notre compréhensions vis à vis du travail à réaliser et des fonctionnalitées recherchées.
Ainsi nous pouvions répondre du mieux possible aux attentes et fournir un site web corrrespondant à ce qui est recherché.
Naturellement nous avons opté pour le langage UML. On peut ainsi représenter et synthétiser les principales fonctions.

Nous avons opté pour l'utilisation des diagrammes statiques de cas d'utilisation. Ils mettent en oeuvre pour chaque type d'utilisateur les différentes actions qu'ils peuvent réaliser sans pour autant donner d'informations précises sur la façon de les implémenter.
Nous ne présenterons ici, que les diagrammes principaux. Les autres seront présents en annexe pour des raisons de lisibilité.
Afin de ne pas surcharger nos diagrammes, nous posons la convention suivante: Certain cas d'utilisation peuvent présenter 
dans leur description le symbole suivant: "(++)". Cela signifie que le cas d'utilisation en question est detaillé sur une autre feuille, reprenant l'intégralité du diagramme.

 
    \section{Framework}
Pour la réalisation du site web, nous avions pensé organiser notre code en utilisant de simple scripts.
Ces pages utilisent l'architecture MVC (Modèle Vue contrôleur) afin de bien séparer chaque partie.

Chaque page aurait été accessible depuis l'index, et l'action fournie par l'url donne la page à afficher.
Ce développement ne convenait pas car la taille de l'application était un peu trop conséquente.

Nous avons donc appris davantage de techniques sur les patrons de conception, et avons organisé notre code
grâce à un framework "artisanal". Cette partie a pour but de présenter ce framework reposant sur l'architecture MVC.
Le framework récupère donc depuis l'url l'action du contrôleur et ses paramétres (optionnels) et exécute l'action.
Chaque action appartient donc à un module et existe en tant que route. Si une route est associée à une action existant dans
un module (i.e un contrôleur), alors elle est executée, et la page associée est générée puis envoyée.
Si la route n'existe pas, on est redirigé vers l'erreur 404 que nous avons défini.

N.B. Une partie du framework vient d'un turoriel du Site du Zéro \cite{ref_framework_mvc}, 
où nous avons trouvé un bon nombre de conseils et de bonnes pratiques.

        \subsection{L'architecture MVC}
Avant de décrire le framework que nous avons mis en place, une explication sur le patron de conception Modèle Vue contrôleur
est nécessaire. Il s'agit donc d'une architecture basée sur trois entités distinctes:

    \begin{itemize}
    \item Modèle
    \item Vue
    \item contrôleur
    \end{itemize}

Le modèle représente les données manipulées, principalement en accèdant en lecture et en écriture à la base de données.

La vue représente l'affichage, comment les informations sont agencées sur la page.

Le contrôleur représente la logique de l'application, c'est à dire le traitement des données obtenues par le modèle
qui sont destinées à être affichées par la vue.

L'élément central est donc le contrôleur, qui gère lui même pour chacune de ses actions son modèle et ses vues (plusieurs
manières d'afficher la même information).

        \subsection{Retour sur le framework}

La première classe, l'Application représente le pont entre tous ses composants. Il existe donc deux applications complémentaires, une pour les utilisateurs, et une pour les admins.
L'application contient les requêtes et les réponses HTTP, donnant ainsi les informations que l'utilisateur envoie et reçoit.

La classe HTTPRequest contient les variables GET et POST, les cookies ainsi que l'url entrée par l'utilisateur.
La classe HTTPResponse contient quant à elle la page associée à la réponse, et permet l'ajout de cookies, l'ajout d'entêtes
ainsi que l'exécution de la page, ou une redirection.

Les ApplicationComponants, constituent une partie de l'application, les blocs de base sur lequel repose le framework.
Chaque composant dispose d'une référence vers l'Application, ainsi, chaque composant est à même d'échanger avec les autres.
On retrouve bien la notion de pont pour l'Application.

La page elle même est composée d'un template et d'une vue générée.
Le template contient principalement le design général du site, le header, le menu ainsi que le footer.
La vue contient les informations relatives à la route choisie, c'est la partie la plus importante.

Une Page dérive également de l'ApplicationComponant, et permet donc de passer des variables depuis le contrôleur à la vue,
cette classe contient également le nom de la vue associée.

Il ne reste plus qu'à construire les contrôleurs de chaque module.
Afin de continuer avec l'approche orientée objet, on définit la classe BackController, qui permet d'être
suffisamment générique pour être commune à chaque futur contrôleur.

Le BackController contient donc les informations suivantes, obtenues depuis la route:

    \begin{itemize}
    \item Nom du module
    \item Nom de l'action
    \item Nom de la vue
    \item Référence vers la Page
    \end{itemize}

Chaque contrôleur est donc une classe dérivant du Backcontrôleur (qui lui même a accès aux autres portions de l'application).
Cela permet d'atteindre les autres parties du système et intéragir avec leurs utilisateurs.

Chaque action correspond à une méthode appartenant à la classe, et cette action est executée à l'appel de la page.
Le BackController se charge de récupérer la vue et de dire la Page de se générer. On peut ensuite utiliser HTTPResponse pour
renvoyer la Page à l'utilisateur.

Le modèle correspond aux données manipulées, et stockées en base de données. On accède en lecture et en écriture à la base de données
via des managers, qui permettent ainsi de manipuler la partie Modèle de l'architecture.

Les vues sont de simples fichiers PHP comprenant principalement du code XHTML. En effet, le but de ce fichier est de spécifier
l'affichage des informations du modèle, et manipulées dans le contrôleur. On y trouve un peu de scripting en PHP, surtout
pour y faire des affichages conditionnels, ainsi que des boucles pour afficher une séries d'informations liées.

Le User est une partie de l'application, il est donc dérivé de l'ApplicationComponant. Ici, le User identifie comme on peut
s'y attendre chaque utilisateur. L'utilisateur peut avoir ses propres attributs, et l'on détermine simplement s'il est authentifié
ou non. On peut associer un message flash (qui n'apparaît qu'une fois sur la page renvoyée à l'utilisateur), et qui est donc personnel
à l'utilisateur en cours.

Présenté un peu plus tôt, les managers sont essentiels pour manipuler les données de l'application. On crée donc une classe
Managers permettant d'exploiter la généricité sur chaque Manager spécifique que l'on va utiliser. Bien entendu, la classe
Managers dérive de l'ApplicationComponant afin d'être accessible principalement pour les contrôleurs spécifiques.

Chaque Manager manipule donc les classes "entités", ce sont juste une encapsulation de chaque information. Cela signifie
qu'un enregistrement dans la base de données correspond généralement à une instance d'une classe. Il y a bien entendu quelques
informations que nous ne stockons pas dans notre propre BDD (Base de données), mais que nous tirons de la BDD de Polytech.

Le Router est une classe qui dérive de l'ApplicationComponant, et qui récupère dans le contrôleur associé.
Une route s'écrit de la manière suivante:

    \lstset{language=XML}
    \begin{lstlisting} 
<route url="/lectures/show-([0-9]+)\.html"
       module="lectures"
       action="show"
       vars="idPackage"/>
    \end{lstlisting}

L'url du fichier contrôleur est indiquée, chaque contrôleur dispose d'un module associé, d'un nom d'action existant pour ce
contrôleur. On voit également qu'il est possible de fournir des valeurs dans l'url. Ces valeurs seront récupérées afin
d'être accessible en données GET. On peut mettre des conditions sur ces variables afin qu'elles respectent l'expression régulière
utilisée.

Nous utilisons également une fonctionnalité des serveurs apache, appelé URL-Rewritting. Cela consiste à ce que l'utilisateur
puisse entrer une certaine url, et que le serveur en reçoive une différente, qui sera donc traitée spécifiquement par le framework.
C'est ici que les routes interviennent, elles font le lien entre l'url de l'utilisateur et l'url donnée au serveur.

Le point fort de l'URL-Rewriting, et donc des routes, et de pouvoir assurer l'existence des pages liées au site web,
ainsi que de tester la validité des paramètres de la requête HTTP.

        \subsection{Classes annexes}

En plus du framework, nous avons créé quelques classes permettant de manipuler plus facilement, et de manière objet
des éléments revenant souvent dans le projet. En effet, nous utilisons de nombreuses fois des dates, et des horaires.

Nous avons donc une classe Date et une classe Time, permettant de créer des dates à partir de chaines de caractères,
on teste ainsi leur validité grâce à une expression régulière. On dispose également de primitives permettant de
faire des comparaisons entre des dates ou des heures. Le format est également adapté pour fonctionner avec MySQL.

Une classe permet également de générer automatiquement les formulaires utilisés, afin d'éviter au maximum les erreurs,
et de simplifier la lecture des vues.

%TODO: class Database William


    \section{Base de données}
    Afin de centraliser toutes les données gravitant autour du site web, nous avons
mis en place une base de données. Comme celle-ci est vouée à être déployée sur le 
serveur de Polytech, nous avons utilisé le même moteur SQL que ce dernier, MySQL.

Toute l'architecture de la base de données est disponible dans l'annexe correspondante.

    \section{Style graphique}
Le fonctionnement du site est une chose importante, mais l'utilisateur ne voit que
la partie émergente de l'iceberg, autrement dis, le design du site web. Il est donc
important de faire en sorte que le site soit plaisant aux yeux à ses yeux. 

Malheuresement, aucun de nous ne posséde une âme d'artiste, et nous avons donc eu
recours à deux camarades de classe, bien plus expérimentés dans ce domaine: Kevin
Langles et Kevin Yot.

La photo ci-dessous correspond au résultat: une charte graphique avec une dominante 
bleu, rapellant les couleurs de Polytech, nuancée par une touche de blanc.

    \begin{figure}[h]
        \begin{center}
        \includegraphics[scale=0.4]{images/screenSiteWeb.png} 
        \end{center}
        \caption{Design du site web}
        \label{Design du site web}
    \end{figure}
\newpage


    \section{Règles d'utilisation}

Afin d'exploiter au mieux l'interface Web du côté administrateur, il est primordial
de suivre la timeline, cela permet ainsi d'être sûr de ne pas tomber sur les "effets de bord"
liés à l'implémentation.

        \subsection{Check list}

La timeline suit le schéma suivant:

    \begin{itemize}
    \item Création des packages ainsi que les conférences, il est \emph{très important} que
           les dates et les horaires soient fixés, et qu'il ne change plus à partir du début des
           inscriptions des étudiants.
    \item Création des séances de QCM pour les promotions participant à la Semaine du Numérique,
           en n'hésitant pas à y mettre des dates distantes, qui pourront être modifiées plus tard.
    \item Envoi d'un mail récapitulatif, rappelant aux étudiants n'ayant pas fait tous leurs choix
           qu'ils seront répartis automatiquement.
    \item Verrouillage \emph{manuel} des inscriptions à la date limite.
    \item Affectation des étudiants ayant des inscriptions manquantes.
    \item Changement des dates des séances de QCM si besoin.
    \item Récupération des notes.
    \end{itemize}

        \subsection{Recommandations}

Pour le déploiement du site web, les pré-requis sont les suivants:

    \begin{itemize}
    \item Serveur Apache2
    \item Serveur PHP 5.2 (ou supérieur)
    \item Serveur MySQL 5.0 (ou supérieur)
    \item Extensions PHP mysqli, zip
    \item Module URL-Rewriting activé sur le serveur Apache
    \end{itemize}

Le fichier .htaccess à la racine de l'application contient les règles pour
l'URL-Rewriting. Certains serveurs bloquent les fichiers .htaccess, il est alors
nécessaire d'écrire ces règles dans le fichier de configuration d'Apache.

Nous avons suivi l'architecture du serveur Polytech, il faut donc placer le code
dans un répertoire nommé "vbMifare/" situé à la racine du serveur Apache.

\section{PDFBox}

\begin{frame}{PDFBox}
  	\begin{block}{Visionneuse}
  		 \begin{itemize}
	   		\item Visionneuse de documents
      		\item Upload zip contenant les diapositives (format image)
	 	\end{itemize}
  	\end{block}

	Comment convertir un pdf en image?

	\begin{block}{Première approche}
  		\begin{itemize}
      			\item Parsing PDF
      			\item jpedal
      			\item iText
  		\end{itemize}
	\end{block}
\end{frame}

\begin{frame}{PDFBox}
	\begin{block}{Présentation}
		\begin{itemize}
			\item Projet Open Source (Apache Software Foundation)
			\item Java 
			\item Conversion PDF -$>$ Image
			\item Limitation: Police Arial
		\end{itemize}
	\end{block}

	\begin{block}{Problèmes}
		\begin{itemize}
			\item Redimensionnement
			\item Décalage
			\item Mauvais caractères imprimés
		\end{itemize}
	\end{block}
\end{frame}

\begin{frame}{fonctionnement}
	%\begin{block}{e}	
		\begin{figure}[h]
        			\begin{center}
         			\includegraphics[scale=0.30]{images/pdfboxTraitement.png} 
        			\end{center}
        			\caption{PDFToImage}
        			\label{PDFToImage}
    		\end{figure}
	%\end{block}
\end{frame}

\begin{frame}{Problème: cropbox/mediabox}
	\begin{block}{Redimensionnement}
		\begin{figure}[h]
        		\begin{center}
         		\includegraphics[scale=0.08]{images/fail1.jpg} 
        		\end{center}
        		\caption{erreur}
        		\label{erreur}
    		\end{figure}
	\end{block}
\end{frame}

\begin{frame}{Polices}

	\begin{block}{Mauvaise gestion des polices}
		\begin{figure}[h]
        		\begin{center}
         		\includegraphics[scale=0.17]{images/fail2.jpg} 
        		\end{center}
        		\caption{Erreur}
        		\label{Erreur}
    		\end{figure}
	\end{block}
\end{frame}

\begin{frame}{}
Le problème vient de:
\begin{itemize}
\item Police incluse et "subset"
\item Manque certaines tables (exemple: name)
\end{itemize}	
\end{frame}

\begin{frame}{Polices}
	\begin{block}{mauvais gestion des polices}
		\begin{figure}[h]
        		\begin{center}
         		\includegraphics[scale=0.17]{images/succes.jpg} 
        		\end{center}
        		\caption{Erreur}
        		\label{Erreur}
    		\end{figure}
	\end{block}
\end{frame}

\chapter{Lecteur de cartes}
Afin de vérifier l'assiduité des élèves lors des conférences nous mettons en
place un système de badge.

Il nous fallait une solution alliant plusieurs critères comme un coût et temps
de développement réduit et de la fabilité.

\newpage

\section{Historique}
Dès le début du projet Monsieur Berry nous a proposé plusieurs pistes à 
explorer pour le badge en lui même. Il fallait que chaque participant aux
conférences puisse être identifié de manière unique:

\begin{itemize}
\item Money Kart: C'est une entreprise basée près de Grenoble, spécialisée
dans la création de cartes à puce, notamment Moneo. L'avantage de cette solution
résidait dans la design des cartes. On aurait pu les faire graver
de façon à faire un buzz autour de la Semaine du Numérique. De plus la société
vendait également des lecteurs prêts à l'emploi.
Malheuresement, il nous fût impossible d'obtenir un devis de la part d'un
commercial, même après de multiples relance. Cette solution nécéssitait aussi
un investissement financier.

\item Carte étudiantes: Rapidement cette solution à fait l'unanimité. En effet,
elle était bien plus facile à mettre en place: chaque titulaire de carte délivrée
par l'UM2 possède un numéro unique, appelé numéro Mifare, accessible grâce à
la technologie RFID. Or Polytech possède une base de donnée faisant le lien entre
ce numéro et les élèves.
\end{itemize}

\section{RFID et Mifare}
RFID, de l'anglais ``radio frequency identification'' est une techonlogie 
mise au point pour permettre de lire à distance des données contenues dans des
marqueurs, étiquettes RFID.

    \begin{figure}[h]
        \begin{center}
            \includegraphics[scale=0.6]{RFIDtag.jpg} 
        \end{center}

        \caption{Etiquette RFID}
        \label{Etiquette RFID}
    \end{figure}


C'est une technologie employée quotidiennement par tout le monde puisqu'elle
sert entre autre à:

\begin{itemize}
\item Passeport bimométrique français.
\item Accès transport public, comme le tramway de Montpellier.
\item Inventaires.
\end{itemize}

Mifare est une des technologies de carte à puce sans contact les plus répandues
dans le monde avec 3,5 milliards de cartes et 40 millions de modules de lecture/encodage.

Ces deux technologies respectent des normes ISO.

\section{Cahier des charges}
%demander a catebras et tanby si les citer ne les dérange pas
Avant tout, la décision de mise en place d'un lecteur de carte correspond au désire d'un gain 
de temps et de facilité d'utilisation. De plus cela revêt un côté écologique dans le sens 
où la consommation de papier ne sera pas afféctée par les fiches d'appels. Mr G.Catebras en 
collaboration avec Mr J.Tanby et V.Berry ont décidé de mettre en place le lecteur pour palier aux
 défauts du lecteur (MFR120U).\newline
\subsection{problème posé}
Le module MFR120U remplissait sa fonction principal mais doit être substitué à cause des défauts suivant:
\begin{itemize}
\item pour avertir qu'une carte a bien été badgée, le module doit pouvoir générer un signal lumineux clairement visible. Le MFR120U possède une led mais elle est cachée car la carte de l'étudiant lors du badgage. Un signal sonore étant à bannir dans les salle de cours ou même dans un amphithéatre.
\item sa taille trop petite et sa forme.
\end{itemize}
De ce fait nous avons mis en place un programme de badgage dont le fonctionnement est analoque à celui du MFR120U qui sera utilisé de concert avec le module fournit par Mr J.Tanby que nous avons rencontré pour lui notifier les caractéristiques voulues ainsi que la technologie qu'il devait utiliser pour communiquer en bluethoot, pour que le programme (nous avon commensé le codage de l'application en nous basant sur le MFR120U) soit compatible avec le module qu'il créera et pour minimiser au maximum les modifications à aporter. \newline
%TODO: parler du protocole
\subsection{Expression fonctionnelle du besoin}
Les principales fonctions du programmes sont:
\begin{itemize}
\item 
\end{itemize}
\subsection{Les solutions proposées pour répondre à ce besoin}
\begin{itemize}
\item 
\end{itemize}
\section{Modélisation}
\section{Recommandations}

<<<<<<< HEAD
\chapter{Conclusion}

Le projet lié à la Semaine du Numérique, proposé par Mr Berry à notre demande à la
fin du premier semestre de PEIP2, nous aura permis de progresser sur plusieurs terrains,
notamment en programmation Web, mais également en gestion d'équipe, .

Du point de vue de la programmation, nous avons exploité plusieurs langages et abordé
des thèmes très différents.
On notera surtout l'utilisation d'un framework en PHP afin de structurer au mieux le site web.
Toujours sur le plan technique, nous avons également utilisé des outils liés au développement
informatique tels que CMake ou git.
Notre groupe était plutôt organisé et intéressé, et nous avions chacun des connaissances à
un différent niveau pour chaque domaine étudié, ce qui nous a permis de tous progresser.
Nous avons dû rencontrer diverses personnes pour accomplir certaines tâches tels que l'activation
d'un module dans le fichier de configuration des serveurs de Polytech, ou encore pour discuter des fonctions
que le badger devait intégrer.
Nous aurons aussi appris à travailler en collaboration sur un projet, à se coordonner et
à se partager le travail.

Bien que le projet ne soit pas encore fini, la plupart des fonctionnalités sont opérationnelles.
Le site peut dès aujourd'hui être déployé et aucun problème majeur n'est connu. 
Il reste tout de même un travail de fond dans le code consistant à "nettoyer" le code, ainsi que
l'implémentation de certaines fonctionnalités comme la génération de statistiques.

Afin de finir le projet, il est donc probable que certains membres du projet réalisent un stage
en collaboration avec le LIRMM, notamment pour le déploiement et la finalisation des fonctionnalités.

=======
\chapter{Environnement}
    Dans ce chapitre nous allons présenter l'environnement dans lequel nous avons
    développé les deux projets. Il va aussi servir à introduire un ensemble de
    règles et de conventions utiliser tout par la suite.

\newpage
    \section{Réunions et contacts}
        Tout au long du projet, nous avons été en contact permanent avec Mr Berry,
    que ce soit par e-mails, appels téléphoniques ou réunions. Ces dernières par
    exemples étaient quasi-hebdomadaires, afin de rester le plus possible dans la
    bonne direction. Nos recontres ont eu lieu à Polytech' ou au Lirmm.


    \section{Gestionnaire de versions}
        Dès le balbutiement du projet, nous avons décidé d'utiliser un gestionnaire
    de version. En effet, celui-ci nous permettait d'avoir un contrôle total des modifications
    tout au long du développement.

    Notre choix s'est naturellement porté sur Git \cite{git}. Cela s'explique 
    par les aspects suivants:

    \begin{itemize}
    \item Multi-plateformes.
    \item Gestions des branches.
    \item Developpement actif et forte communauté d'utilisation.
    \item Rapidité d'éxécution comparé à svn, hg, etc. 
    \end{itemize}

    De plus certains d'entre nous l'avait déjà utilisé précédement: git est assez
    complèxe à prendre en main au début et l'entraide était de mise.

    Mais surtout, le plus important pour nous est le site web GitHub \cite{github}.
    Il fait office de serveur centralisé pour git. Mais pas seulement puisqu'il 
    propose entre autre:

    \begin{itemize}
    \item L'hébergement de projets sous Git.
    \item Des fonctionnalités de type réseaux sociaux, dont :
        \begin{itemize}
        \item Les flux.
        \item Le suivi de personnes ou de projets.
        \item Les graphes de réseau pour les dépôts.
        \end{itemize}
    \item Un pastebin nommé Gist.
    \item Un wiki et une page web pour chaque dépôt.
    \end{itemize}

    C'est un site web qualifié de professionnel puisqu'il est utiliser par de 
    nombreux projets de grandes tailles tels que : Git, Perl, Facebook, Twitter,
    JQuery, PHP, Python, etc.

        De plus Mr Berry, de part le Lirmm, ne pouvait nous ouvrir qu'un serveur
    subserversion, donc Git couplé avec GitHub est un excellent compromis.

    Nota Bene : son adoption sous les systèmes Windows relève encore du portage
    expérimental. Il a été codé par le créateur de Linux, pour versionner le code
    de ce dernier, et s'utilise par conséquent en lignes de commandes.






\newpage
    \section{Règles de codage}
        Ci-dessous, voici les règles de codage utilisées tout au long du projet
    ainsi que dans le rapport. Elles sont là pour permettre une cohérence tout au
    long du développement, mais aussi et surtout, pour que la relecture et la 
    compréhension soient simplifiées pour les développeurs du projet, ainsi que pour les
    futures personnes amenées à travailler dessus.

    Cela regroupe:
    \begin{itemize}
    \item Un code en anglais (commentaire compris).
    \item Eviter de dépasser 80 caractères si possibles.
    \item Indentation de 4 espaces, pas de tabulation.
    \end{itemize}

    Ci-dessous, un fichier C++ d'exemple.
\newline

    \begin{lstlisting}
#ifndef BADGER_FOO_HPP
#define BADGER_FOO_HPP

#include <string>

namespace badger
{
    ///////////////////////////////////
    /// \brief Sample class
    ///
    /// Use for example.
    ///////////////////////////////////
    class FooBar
    {
    public:
        virtual const int& doSomething() const = 0;

    private:
        int m_attribute;
    };

} // namespace badger

#endif //BADGER_FOO_HPP
    \end{lstlisting}





>>>>>>> d9df73bf2f7f56ec84835544ca816bb7ee0d0249

\end{document}

