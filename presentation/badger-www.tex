\section{Site web}

\begin{frame}{Semaine du Numérique}
    \begin{block}{Semaine du Numérique}
    Formation liée à l'informatique pour Polytech' Montpellier
    Demande de la CTI

    Connaissances en informatique:
    \begin{itemize}
    \item Systèmes d’information : ERP
    \item Maîtrise d’ouvrage
    \item Sécurité
    \item Collecticiels
    \item Architecture système
    \item Valeurs juridiques (dépendant des SHEJS)
    \end{itemize}
    \end{block}
\end{frame}

\subsection{Timeline du projet}
\begin{frame}{Timeline du projet}
    Etapes de la Semaine du Numérique:
    \begin{block}{Timeline}
    \begin{itemize}
    \item Mise en place des informations
    \item Inscription des étudiants
    \item Semaine du Numérique
    \item Séances de QCM
    \item Remise de rapports
    \end{itemize}
    \end{block}
\end{frame}

\begin{frame}{Fonctionnement des conférences}
    \begin{block}{Package}
    Ensemble de conférences reliées autour d'un même thème
    \end{block}

    \begin{block}{Inscriptions aux packages}
    Inscription à un nombre fixé de packages (indépendant du nombre de conférences) avant la date limite
    \end{block}
\end{frame}

\begin{frame}{Notation}
    \begin{block}{Critères de notation}
    \begin{itemize}
    \item Note de présence
    \item Evaluation par QCM
    \item Rapports
    \end{itemize}
    \end{block}

    \begin{block}{QCM}
    Questions choisies aléatoirement parmi celles associées aux packages auxquels l'étudiant est inscrit
    \end{block}

    \begin{block}{Rapports}
    Pour chaque package: Rapport synthétisant les informations importantes des conférences
    \end{block}
\end{frame}

\subsection{Cahier des charges}
\begin{frame}{Cahier des charges - Utilisateur}
    \begin{block}{Fonctionnalités}
    \begin{itemize}
    \item Liste des packages
    \item Planning
    \item Téléchargement des documents associés
    \item Visionneuse de documents en ligne
    \item Passe de QCM
    \item Remise de rapports
    \end{itemize}
    \end{block}
\end{frame}

\begin{frame}{Cahier des charges - Administrateur}
    \begin{block}{Fonctionnalités}
    \begin{itemize}
    \item Création / Edition:
            \begin{itemize}
            \item Packages
            \item Conférences
            \item Questions / Réponses
            \item Salles / Disponibilités
            \end{itemize}
    \item Mise en ligne de documents (téléchargement direct et visionneuse)
    \item Inscription de promotions pour les QCM
    \item Remise à zéro
    \end{itemize}
    \end{block}
\end{frame}

\subsection{Framework}
\begin{frame}{Architecture MVC}
    \begin{block}{Architecture MVC}
    \begin{itemize}
    \item Modèle
    \item Vue
    \item Contrôleur
    \end{itemize}
    Abstraction séparant la manipulation de données, l'affichage et la logique de l'application
    \end{block}
\end{frame}

\begin{frame}{Framework}
    \begin{block}{Définition}
    Ensemble d'outils et de composants logiciels organisés conformément à un plan d'architecture et des patterns, formant un squelette de programme.
    \end{block}

    \begin{block}{Avantages}
    Framework orienté objet présenté sur le Site du Zéro
    \begin{itemize}
    \item Organisation du code
    \item Séparation Utilisateur / Administrateur simple
    \item Traitement de formulaires simplifié
    \end{itemize}
    \end{block}
\end{frame}

\begin{frame}{Framework - Aspect technique}
    \begin{block}{Classes}
    \begin{itemize}
    \item Application: Pont entre les différentes instances de Componant
    \item Componant: Composant de l'Application (dispose d'une référence vers l'Application)
    \item Page (dérive de Componant): Encapsulation du template et le corps de page
    \item Controller (dérive de Componant): Encapsulation du template et le corps de page
    \end{itemize}
    \end{block}
\end{frame}
