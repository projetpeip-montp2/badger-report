\section{Environnement}
\begin{frame}{Environnement}
	\begin{block}{Besoins}
        Le projet comportant plus de 11 000 lignes de code (en très grande majorité
        de PHP, C++ et Javascript), il est nécessaire d'avoir un environnement 
        adapté.

	    \begin{itemize}
            \item Gestionnaire de versions.
            \item Licence adapté au contexte.
	    \end{itemize}
	\end{block}
\end{frame}


\subsection{Gestionnaire de version}
\begin{frame}{Git}
	\begin{block}{Git}
        Git est un gestionnaire de versions performant et de nouvelle génération.

        \begin{itemize}
        \item Multi-plateformes.
        \item Gestions des branches.
        \item Développement actif et forte communauté d'utilisation.
        \item Rapidité d'exécution comparé à svn, hg, etc. 
        \end{itemize}
	\end{block}
\end{frame}


\begin{frame}{GitHub}
	\begin{block}{GitHub}
    GitHub est une plateforme libre proposant de gérer les repository Git de 
    façon automatique et centralisé. Les services offerts sont:

        \begin{itemize}
        \item L'hébergement de projets sous Git
        \item Des fonctionnalités de type réseaux sociaux, dont :
            \begin{itemize}
            \item Les flux
            \item Le suivi de personnes ou de projets
            \item Les graphes de réseau pour les dépôts
            \end{itemize}
        \item Un pastebin nommé Gist
        \item Un wiki et une page web pour chaque dépôt
        \end{itemize}
    \end{block}
\end{frame}


\subsection{Licence}
\begin{frame}{Licence}
	\begin{block}{Licence BSD}
        Tout le code présenté est sous licence BSD afin de permettre de pouvoir
        être développé et continué par d'autres personnes aux sans quelques
        problèmes que ce soit.
	\end{block}
\end{frame}

