\chapter{Site web}
    \section{Timeline de la Semaine du Numérique}

    Afin de présenter au mieux les objectifs du site Web, il est important de comprendre comment
    est censé se dérouler la Semaine du Numérique.

    ... Description

    Les packages (appelés aussi cycles de conférences) correspondent à un ensemble de conférences regroupées autour d'un même thème.
    Chaque étudiant peut donc choisir parmi le panel disponible, c'est donc un système "à la carte", permettant
    ainsi de personnaliser cette unité d'enseignement.

    Une fois les conférences passées, les étudiants doivent passer un QCM généré en fonction des packages qu'ils ont choisi.
    Les étudiants doivent également rendre un rapport pour chaque package auxquels ils se sont inscrits.

    \section{Cahier des charges}
    Le site Web que nous avons développé est une plateforme de communication entre les étudiants
    et Mr Berry, en charge de la Semaine du Numérique (ainsi que de notre projet).
    L'application devait donc fournir un certain nombre de fonctionnalités vis à vis des étudiants,
    mais aussi pour l'administration, afin de gérer les potentiels conflits.

    Du côté utilisateurs, (également appelé Frontend), les étudiants ont accès aux pages suivantes:

    Les packages disposent d'une courte description présentant le thème, ainsi qu'une liste de conférences.
    Ces conférences ont également une conférence, ainsi que le nom (et parfois la société) du conférencier,
    la date, l'horaire et la salle où celle-ci aura lieu.
    Les étudiants ont donc accès à la liste des packages, et peuvent s'inscrire et se désinscrire
    d'un package à tout moment dans la limite des inscriptions.

    Les étudiants peuvent voir le planning des conférences auxquels ils sont inscrits

    \begin{itemize}
    \item Inscription / Désinscription d'un package de conférences
    \item Accès à un planning des packages sélectionnés
    \item Passage d'un QCM généré en fonction des choix étudiants
    \item Dépôt de rapports sur le serveur
    \end{itemize}    

    \section{Modélisation}
    \section{Framework}
        \cite{ref_framework_mvc}

    \section{Check list}
    \section{Recommandations}

