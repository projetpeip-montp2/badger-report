\chapter{Lecteur de cartes}

Afin de vérifier l'assiduité des élèves lors des conférences, nous mettons en
place un système de badge.

Il nous fallait une solution alliant plusieurs critères comme un coût limité, un temps
de développement réduit et de la fiabilité.


    \section{Historique}
Dès le début du projet, Monsieur Berry nous a proposé plusieurs pistes à 
explorer pour le badge en lui-même. Il fallait que chaque participant aux
conférences puisse être identifié de manière unique:

\begin{itemize}
\item Money Kart: C'est une entreprise basée près de Grenoble, spécialisée
dans la création de cartes à puce, notamment Moneo. L'avantage de cette solution
résidait dans la design des cartes. On aurait pu les faire graver
de façon à faire un buzz autour de la Semaine du Numérique. De plus la société
vendait également des lecteurs prêts à l'emploi.
Malheureusement, il nous fut impossible d'obtenir un devis de la part d'un
commercial, même après de multiples relance.

\item Carte étudiant: Rapidement cette solution a fait l'unanimité. En effet,
elle était bien plus facile à mettre en place: chaque titulaire de carte délivrée
par l'UM2 possède un numéro unique, appelé numéro Mifare, accessible grâce à
la technologie RFID. Or Polytech possède une base de données faisant le lien entre
ce numéro et les élèves.
\end{itemize}


    \section{RFID et Mifare}
RFID, de l'anglais ``radio frequency identification'' est une technologie 
mise au point pour permettre de lire à distance des données contenues dans des
marqueurs, les étiquettes RFID.

    \begin{figure}[h]
        \begin{center}
            \includegraphics[scale=0.3]{images/RFIDtag.jpg} 
        \end{center}

        \caption{Etiquette RFID}
        \label{Etiquette RFID}
    \end{figure}


C'est une technologie employée quotidiennement par tout le monde puisqu'elle
est utilisée dans les cas suivants:

    \begin{itemize}
    \item Passeport bimométrique français
    \item Accès transport public, comme le tramway de Montpellier
    \item Inventaires
    \end{itemize}

Mifare est une des technologies de carte à puce sans contact les plus répandues
dans le monde avec 3,5 milliards de cartes et 40 millions de modules de lecture/encodage.

Ces deux technologies respectent les normes ISO.


    \section{MFR120U}
Rapidement après le début du projet, Mr Berry nous a mis en contact avec Mr
Cathebras, directeur du département ERII à Polytech'. En effet, il souhaitait lui
aussi mettre en place un système de présence informatisé, pour éviter de devoir
le faire sur papier, avant de le reporter sur l'ordinateur.

Or il avait déjà entramé les recherches de son côté, puisqu'il avait acheté un lecteur de carte 
Mifare, appelé module MFR120U, pour la somme de 120\euro. Ce lecteur présentait néanmoins
plusieurs inconvénients majeurs:

    \begin{itemize}
        \item Programme d'utilisation exclusivement sous les plateformes Windows.
        \item Liaison série-USB avec l'ordinateur par câble, ce qui nécessite 
              à l'utilisateur d'installer un driver fournis également que pour
              Windows.
        \item Difficulté pour savoir si la lecture de la carte avait réussi ou non.
    \end{itemize}

Par contre il possède un écran LCD, ce qui peut se révéler assez pratique pendant

Mr Cathebras avait donc demandé à un de ses collègues de travail du département
ERII, Mr Tamby, s'il pouvait concevoir un module qui répondrait plus à ses attentes.
De notre côté, nous devions nous occuper de toute la partie software.

C'est ici que nous intervenons. Début février 2012 nous avons recontré Mr Cathebras,
et nous avons pu discuter de l'intérêt commun à développer un tel lecteur. En effet,
s'il venait à faire ses preuves en plus lors des conférences, il pourrait se voir
industrialisé afin de voir son utilisation généralisée à Polytech. Il nous a donc
confié le MFR120U pour que nous puissions nous faire la main dessus.

    \begin{figure}[h]
        \begin{center}
            \includegraphics[scale=0.55]{images/mfr.jpg}
        \end{center}
        \caption{MFR120U}
        \label{MFR120U}
     \end{figure}  

De plus il nous a fourni la documentation du lecteur. Or celle-ci nous fut d'un
grand secours par la suite puisqu'elle comprenait tout le protocole de communication
utilisé par le lecteur, ainsi que de nombreuses autres caractéristiques.

    \begin{itemize}
    \item Fabricant de la puce: Prolific Technology Inc. Les drivers pour Windows
          et MacOS X sont disponibles sur leur site web \cite{prolific}
    \item Baud rate: 19200
    \item Communication série 8N1 : 8 bits de donnés, aucun de parité, et 1 de stop
    \item Possibilité de lire les cartes Mifare, ainsi que d'autres normes
    \item En interne, le lecteur possède une mémoire capable de stocker 8000
          lectures de badge (date, heure, et informations contenues sur la carte)
    \end{itemize}

En interne, la gestion des enregistrements fonctionne grâce à un curseur.
De base, il pointe sur le premier enregistrement. On ne peut que demander le suivant,
ce qui a pour effet d'incrémenter ce curseur et de renvoyer la nouvelle valeur
pointée, ou faire un ``rollback'', c'est à dire remettre le curseur sur le tout
premier enregistrement.

Le programme que nous devions développer était en somme assez simple, puisqu'il
devait offrir une interface permettant de se connecter au lecteur, de récupérer les
données contenues suivant certains critères, les supprimer en mémoire si nécessaire,
et enfin les envoyer sur une base de données distante.

La version fonctionnelle pour le MFR120U (sans l'envoi en base de données)
est disponible dans le repository git sous le tag v0.9-mfr120u.


    \section{Caractère multi-plateformes}
Afin de répondre à la demande multi-plateforme des professeurs, nous nous 
sommes naturellement tourné vers le language C++. En effet, celui-ci offre un accès
proche de la machine, intéressant pour la communication série notamment. De plus,
un étudiant ERII 3, Arthur Hiairrassary, a mis à notre disposition la bibliothèque
pour faire transiter des informations séries.

Par contre, nous avons développé avec le nouveau standard C++11. Ainsi
dans le code nous trouvons l'utilisation des concepts suivants:

    \begin{itemize}
    \item Variadics templates
    \item Fonctions lambdas
    \item Rvalues references
    \item Move-constructor
    \end{itemize}

L'inconvénient de cela est bien sur le support par les compilateurs de cette
norme. Par exemple, msvc, le compilateur de Microsoft, ne prend en charge que peu
de ces fonctionnalités, et n'est donc pas à même de compiler le code, même avec la
beta de Visual Studio 2011.

Néanmoins, l'utilisation de ce standard nous a permis de s'affranchir de nombreuses
restrictions, et par conséquent un développement plus rapide, mais surtout bien
plus sûr. En effet la S(T)L est bien plus avancée.

De plus, l'inconvénient dû au compilateur msvc fut de courte durée puisqu'il
existe g++ et clang++, qui eux supportent en grande partie le nouveau standard.
Et en plus ils sont multi-plateforme! Le programme a pour l'instant été testé 
avec succès dans les conditions suivantes:

    \begin{itemize}
    \item Windows 7, 64 bits, MinGW 4.6.1
    \item MacOS X 10.7, 64 bits, clang++ 3.0 (sous machine virtuelle)
    \item Linux >= 3.1, 64 bits, g++-4.6
    \end{itemize}

Afin de permettre une compilation aisée sur toutes ces plateformes, nous avons
opté pour l'utilisation de CMake afin de générer l'ensemble de l'environnement de
façon automatique.


    \section{Prototype}
Début décembre, nous avons pu rencontrer Mr Tamby, en charge de développer le
prototype du nouveau lecteur. Durant notre première réunion, il nous a donné
bon nombre de renseignements et conseils. De notre côté, nous lui avons fourni
l'ensemble des commandes que nous souhaitions retrouver dans le prototype final.

Aux alentours de mi-avril, il nous a invité à voir le premier prototype.

    \begin{figure}[h]
        \begin{center}
            \includegraphics[scale=0.75]{images/proto.jpg} 
        \end{center}
        \caption{Prototype du lecteur}
        \label{Prototype du lecteur}
     \end{figure} 

Nous avons alors été très surpris par le respect dont Mr Tamby a fait preuve
durant la réalisation du module, tant au niveau électronique que firmware.
En effet il n'aura fallu changer que le baud rate, ainsi que quelques lignes
pour que le programme soit opérationnel pour ce lecteur !

De plus, plus besoin de câble, puisque le prototype fonctionne par Bluetooth,
conformément aux recommandations de Mr Cathebras. Cela engendrera aussi la disparition
du système de droit d'accès puisque pour pouvoir se connecter à un module Bluetooth,
il est possible de spécifier un code PIN.

Le firmware du lecteur est un mélange des langages C et assembleur, pour un total
d'environ 1700 lignes de code. Ci-dessous des brides du code, afin de montrer
que celui-ci reste très proche de la machine.

    \lstset{language=C}
    \begin{lstlisting} 
// Choix du sens des ports I/O
    // Registre OPTION_REG relatif aux interruptions
    OPTION_REG = 0b10000000;

    // Registre INTCON relatif aux interruptions
    INTCON = 0b11000000;
    ANSELH=0;

    // Detecter l'interruption sur RB4
    IOCB =0b00010000;
    \end{lstlisting}


    \lstset{language=[x86masm]Assembler}
    \begin{lstlisting} 
#asm

inter_suite:
	movwf	saveW		// sauvegarde du registre W	
	movf	STATUS,0
	movwf	savesta		// sauvegarde du registre STATUS
	movf    PCLATH,0
	movwf	savePCLATH

	bcf		PCLATH,4
	bcf		PCLATH,3

	bcf	    STATUS,RP0

// ...
    \end{lstlisting}


    \section{Protocole de communication}
Le protocole de communication utilisé n'est pas difficile. Il est orienté 
texte, puisqu'il repose sur le jeu de caractères ASCII, soit 8 bits pour chacun des
caractères, ce qui correspond sur les processeurs Intel en C++ au type char. 
Toute commande s'effectue de la sorte:

STX Command [Data] CR

Les caractères STX et CR, sont non-imprimables, prenant respectivement les
valeurs 0x02 et 0x0D. ``Commande'' est codée sur un seul caractère, alors que ``Data''
est optionnelle et dépend de la commande. Pour le programme du prototype nous avons
entre autres:

    \begin{itemize}
    \item N: Retourne le nombre d'enregistrements
    \item E: Efface la mémoire des enregistrements
    \item T: Retourne la date du lecteur
    \end{itemize}

En retour à chacune des commandes, le lecteur renvoie une séquence, de la forme:

STX ReturnValue [Data] CR

En général, ``ReturnValue'' est le caractère A, 0x41, pour dire que
la commande s'est bien déroulé. ``Data'' est une nouvelle fois optionnelle.

La liste détaillée des commandes du protocole pour le prototype est disponible
en annexe du rapport. 

Attention, elles diffèrent quelque peu de celles du MFR120U, présentes sur 
la documentation officielle.
\newpage

    \section{Architecture matérielle}
Ci-dessous se trouve un schéma représentant le prototype du lecteur de carte.
Nous n'entrerons pas dans les détails puisque cela requiert certaines notions 
d'électroniques.

    \begin{figure}[h]
        \begin{center}
            \includegraphics[scale=0.7]{images/protoSchema.png} 
        \end{center}
        \caption{Schéma du prototype}
        \label{Schéma du prototype}
     \end{figure} 


    \section{Architecture logicielle}
Le programme ``badger'' est décomposé en plusieurs dossier. 

Dans le dossier serial/ on trouve toute la bibliothèque concernant la communication
série. Elle utilise l'idiome de programmation Pimpl (Private Implementation), afin 
de séparer interface et implémentation. Cela est dû au fait qu'elle est implémentée
pour les systèmes POSIX d'un côté (Linux et MacOS), et Win32 de l'autre. Ces derniers
ne supportent pas de façon optimale la norme POSIX.

Le répertoire console/ contient quant à lui tout le système d'interpréteur.
Il est issu d'un projet de moteur de rendu 3D d'un des membres du projet.

Enfin le dossier badger/ contient le programme final, faisant appel à ceux
ci-dessus. Il contient de plus une classe permettant d'envoyer des requêtes SQL
à une base de données MySQL distante. Le programme fonctionne sous la forme d'un 
interpréteur de commandes.
\newpage

    \section{Utilisation}
Ci-dessous, voici l'exemple d'utilisation sous Linux, une fois placé dans le
répertoire du binaire du programme badger:

    \begin{lstlisting}
./badger

open /dev/rfcomm0
count
send passwd 25-12-2011 10:00:00 17:00:00
erase
close
exit
    \end{lstlisting}

L'exemple ci-dessus, ouvre une connexion série sur le descripteur de fichier
/dev/rfcomm0. La commande ``count'' renvoie le nombre d'enregistrement contenus sur 
le module, alors que ``send'' envoie automatiquement dans une base de données (nommée
dans le code) tout les enregistrements effectués le 25-12-2011 entre 10H et 17H.
Enuiste ``erase'' vide la mémoire, ``exit'' ferme la connexion s'il elle existe, et
quitte le programme.

L'ensemble des commandes pour l'interpréteur est aussi disponible en annexe.


    \section{Recommandations et limites}
Actuellement, le plus gros manque dans ce programme (sans considérer une IHM) est
le manque de détection automatique du lecteur: en paramètre de la commande ``open''
l'utilisateur doit spécifier le descripteur de fichier, qui change d'une plateforme
à l'autre, voire de la version de l'OS employé ou du nombre de périphériques connectés:

    \begin{itemize}
    \item Windows: COMx, où x >= 1.
    \item MacOS X: /dev/tty.Polytech-ERII
    \item Linux: /dev/rfcommx, où x >= 0.
    \end{itemize}

Donc nous sommes encore dans la recherche d'une solution ``viable''. Par
rétro-ingéniérie grâce à un programme écoutant les ports séries; nous avons
trouvé comment le programme officiel reconnaissait si le lecteur MFR120U était 
connecté ou pas: au lancement le programme envoie sur tous les ports la commande
STX F CR, qui si elle est envoyé sur le MFR120U, retourne un résultat attendu.

Or cette façon nous dérange, dans le sens où l'on ne sait pas sur quel périphérique
nous allons envoyer cette commande, et encore moins quel va être sa réaction...
