\chapter{Commandes de l'interpréteur}

    Ci-dessous l'ensemble des commandes utilisables dans l'interpréteur. Remarque:
les dates sont au format JJ-MM-AAAA, et les heures au format HH-MM-SS.

\begin{table}[h]
\begin{center}

    \begin{tabular}{|c|M{2.5cm}|M{6cm}|}
    \hline
    Commande & Paramètre(s) & Résultat \tabularnewline
    \hline
    ? &  & Renvoie la liste des commandes possibles. \tabularnewline
    \hline
    open & filename & Ouvre une connexion série sur filename \tabularnewline
    \hline
    close &  & Ferme la connexion si elle existe \tabularnewline
    \hline
    next &  &  Renvoie l'enregistrement suivant \tabularnewline
    \hline
    rollback &  & Remet le curseur à zéro \tabularnewline
    \hline
    erase &  & Efface les enregistrements (avec confirmation) \tabularnewline
    \hline
    count &  & Renoie le nombre d'enregistrements \tabularnewline
    \hline
    getDateTime &  & Renvoie la date \tabularnewline
    \hline
    setDateTime & date time & Change la date \tabularnewline
    \hline
    display & date startTime endTime & Affiche les enregistrements suivants les critères. \tabularnewline
    \hline
    send & passwd Date startTime endTime & Affiche les enregistrements suivants les critères. \tabularnewline
    \hline
    quit &  & Ferme le programme \tabularnewline
    \hline
    exit &  & Alias de exit \tabularnewline
    \hline
    \end{tabular}

\end{center}
\caption{Commandes de l'interpréteur}
\label{Commandes de l'interpréteur}
\end{table}

