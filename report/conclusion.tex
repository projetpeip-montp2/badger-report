\chapter{Conclusion}

Le projet lié à la Semaine du Numérique, proposé par Mr Berry à notre demande à la
fin du premier semestre de PEIP2, nous aura permis de progresser sur plusieurs terrains,
notamment en programmation Web, mais également en gestion d'équipe, .

Du point de vue de la programmation, nous avons exploité plusieurs langages et abordé
des thèmes très différents.
On notera surtout l'utilisation d'un framework en PHP afin de structurer au mieux le site web.
Toujours sur le plan technique, nous avons également utilisé des outils liés au développement
informatique tels que CMake ou git.
Notre groupe était plutôt organisé et intéressé, et nous avions chacun des connaissances à
un différent niveau pour chaque domaine étudié, ce qui nous a permis de tous progresser.
Nous avons dû rencontrer diverses personnes pour accomplir certaines tâches tels que l'activation
d'un module dans le fichier de configuration des serveurs de Polytech, ou encore pour discuter des fonctions
que le badger devait intégrer.
Nous aurons aussi appris à travailler en collaboration sur un projet, à se coordonner et
à se partager le travail.

Bien que le projet ne soit pas encore fini, la plupart des fonctionnalités sont opérationnelles.
Le site peut dès aujourd'hui être déployé et aucun problème majeur n'est connu. 
Il reste tout de même un travail de fond dans le code consistant à "nettoyer" le code, ainsi que
l'implémentation de certaines fonctionnalités comme la génération de statistiques.

Afin de finir le projet, il est donc probable que certains membres du projet réalisent un stage
en collaboration avec le LIRMM, notamment pour le déploiement et la finalisation des fonctionnalités.
