\chapter{Conclusion}

Travailler sur un projet de cette envergure, nous aura permis de mettre à 
profis nos connaissances et de les améliorer. Que ce soit dans le domaine de la 
programmation avec la gestion de la communication avec le lecteur de carte, ou 
alors avec la manipulation de bases de données.
Nous aurons aussi appris à travailler en collaboration sur le même projet et cela de façon
simultané via git.
Dans l'ensemble, le notre groupe était plûtot organisé et intéréssé. Chacun 
mettait ses compétences à profit pour les autres. En cas de problème, l'entraide était de 
mise.
Nous avons du rencontrer diverses personnes pour accomplir certaines tâches tel que l'activation
d'un module dans le fichier de configuration des serveurs de Polytech, ou encore pour discuter des fonctions
que le badger devait intégrer.
Bien que le projet ne soit pas encore fini, la plupart des fonctionnalitées sont 
opérationnelles. Le site peut dès aujourd'hui être déployé et aucun bug majeur
 n'est connu. 
Il reste tout de même un gros travail de fond dans le code consistant à enlever toutes les fonctions 
qui ne sont plus utilisées et ajouter les fonctionnalités manquantes. De plus, la gestion des suppressions en bases de données 
nécéssite d'être améliorée pour gérer dans certains cas la suppression en cascade.


