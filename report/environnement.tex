\chapter{Environnement}
    Dans ce chapitre, nous présenterons l'environnement dans lequel nous avons
    développé les deux projets. Il va aussi servir à introduire un ensemble de
    règles et de conventions utilisées tout par la suite.

\newpage
    \section{Réunions et contacts}
        Tout au long du projet, nous avons été en contact permanent avec Mr Berry,
    que ce soit par e-mails, appels téléphoniques ou réunions. Ces dernières par
    exemples étaient quasi-hebdomadaires, afin de rester le plus possible dans la
    bonne direction. Nos rencontres ont eu lieu à Polytech ou au Lirmm.


    \section{Gestionnaire de versions}
        Dès les premiers balbutiements du projet, nous avons décidé d'utiliser un 
        gestionnaire de version. En effet, celui-ci nous permettait d'avoir un contrôle
        total des modifications tout au long du développement.

    Notre choix s'est naturellement porté sur Git \cite{git}. Cela s'explique 
    par les aspects suivants:

    \begin{itemize}
    \item Multi-plateformes.
    \item Gestions des branches.
    \item Développement actif et forte communauté d'utilisation.
    \item Rapidité d'éxécution comparé à svn, hg, etc. 
    \end{itemize}

    De plus certains d'entre nous l'avait déjà utilisé précédement: git est assez
    complèxe à prendre en main au début cependant l'entraide était de mise.

    Mais surtoût, le plus important pour nousreste quand même le site web GitHub \cite{github}.
    Il fait office de serveur centralisé pour git. Mais pas seulement puisqu'il 
    propose entre autre:

    \begin{itemize}
    \item L'hébergement de projets sous Git.
    \item Des fonctionnalités de type réseaux sociaux, dont :
        \begin{itemize}
        \item Les flux.
        \item Le suivi de personnes ou de projets.
        \item Les graphes de réseau pour les dépôts.
        \end{itemize}
    \item Un pastebin nommé Gist.
    \item Un wiki et une page web pour chaque dépôt.
    \end{itemize}

    C'est un site web qualifié de professionnel puisqu'il est utiliser par de 
    nombreux projets de grandes envergures tels que : Git, Perl, Facebook, Twitter,
    JQuery, PHP, Python, etc.

        De plus Mr Berry, de part le Lirmm, ne pouvait nous ouvrir qu'un serveur
    subserversion, donc Git couplé avec GitHub est un excellent compromis.

    Nota Bene : son adoption sous les systèmes Windows relève encore du portage
    expérimental. Il a été codé par le créateur de Linux, pour versionner le code
    de ce dernier, et s'utilise par conséquent en lignes de commandes.






\newpage
    \section{Règles de codage}
        Ci-dessous, voici les règles de codage utilisées tout au long du projet
    ainsi que dans le rapport. Elles sont là pour permettre une cohérence tout au
    long du développement, mais aussi et surtout, pour que la relecture et la 
    compréhension soient simplifiées pour les développeurs du projet, ainsi que pour les
    futures personnes amenées à travailler dessus.

    Cela regroupe:
    \begin{itemize}
    \item Un code en anglais (commentaire compris).
    \item Eviter de dépasser 80 caractères si possibles.
    \item Indentation de 4 espaces, pas de tabulation.
    \end{itemize}

    Ci-dessous, un fichier C++ d'exemple.
\newline

    \begin{lstlisting}
#ifndef BADGER_FOO_HPP
#define BADGER_FOO_HPP

#include <string>

namespace badger
{
    ///////////////////////////////////
    /// \brief Sample class
    ///
    /// Use for example.
    ///////////////////////////////////
    class FooBar
    {
    public:
        virtual const int& doSomething() const = 0;

    private:
        int m_attribute;
    };

} // namespace badger

#endif //BADGER_FOO_HPP
    \end{lstlisting}




