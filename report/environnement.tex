\chapter{Environnement}
    Dans ce chapitre nous allons présenter l'environnement dans lequel nous avons
    développé les deux projets. Il va aussi servir à introduire un ensemble de
    règles et de conventions utiliser tout par la suite.

\newpage

    \section{Gestionnaire de versions}
        Dès le balbutiement du projet, nous avons décidé d'utiliser un gestionnaire
    de version. En effet, celui-ci nous permettait d'avoir un contrôle total des modifications
    tout au long du développement.

    Notre choix s'est naturellement porté sur Git \cite{git}. Cela s'explique 
    par les aspects suivants:

    \begin{itemize}
    \item Multi-plateformes.
    \item Gestions des branches.
    \item Developpement actif et forte communauté d'utilisation.
    \item Rapidité d'éxécution comparé à svn, hg, etc. 
    \end{itemize}

    De plus certains d'entre nous l'avait déjà utilisé précédement: git est assez
    complèxe à prendre en main au début et l'entraide était de mise.

    Mais surtout, le plus important pour nous est le site web GitHub \cite{github}.
    Il fait office de serveur centralisé pour git. Mais pas seulement puisqu'il 
    propose entre autre:

    \begin{itemize}
    \item L'hébergement de projets sous Git.
    \item Des fonctionnalités de type réseaux sociaux, dont :
        \begin{itemize}
        \item Les flux.
        \item Le suivi de personnes ou de projets.
        \item Les graphes de réseau pour les dépôts.
        \end{itemize}
    \item Un pastebin nommé Gist.
    \item Un wiki et une page web pour chaque dépôt.
    \end{itemize}

    C'est un site web qualifié de professionnel puisqu'il est utiliser par de 
    nombreux projets de grandes tailles tels que : Git, Perl, Facebook, Twitter,
    JQuery, PHP, Python, etc.

    Nota Bene : son adoption sous les systèmes Windows relève encore du portage
    expérimental. Il a été codé par le créateur de Linux, pour versionner le code
    de ce dernier, et s'utilise par conséquent en lignes de commandes.







    \section{Règles de codage}
        Ci-dessous, voici les règles de codage utilisées tout au long du projet
    ainsi que dans le rapport. Elles sont là pour permettre une cohérence tout au
    long du développement, mais aussi et surtout, pour que la relecture et la 
    compréhension soient simplifiées pour les développeurs du projet, ainsi que pour les
    futures personnes amenées à travailler dessus.

    Cela regroupe:
    \begin{itemize}
    \item Un code en anglais (commentaire compris).
    \item Les noms de classe commencent par une majuscule: MaClasse
    \item Les noms de variables: 
    \item Indentation de 4 espaces, pas de tabulation
    \end{itemize}

    Ci-dessous, une classe C++ d'exemple:

    \begin{lstlisting}
    #ifndef BADGER_FOO_HPP
    #define BADGER_FOO_HPP

    namespace badger
    {
        class FooBar
        {

        private:
            int m_;
        };

    } // namespace badger

    #endif //BADGER_FOO_HPP
    \end{lstlisting}




