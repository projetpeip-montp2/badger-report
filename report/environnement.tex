\chapter{Environnement}

Dans ce chapitre, nous allons présenter l'environnement dans lequel nous avons
développé les deux projets. Il va aussi servir à introduire un ensemble de
règles et de conventions utilisées par la suite.
Nous présenterons également les technologies utilisées au cours de ce projet.


    \section{Réunions et contacts}
Tout au long du projet, nous avons été en contact permanent avec Mr Berry,
que ce soit par e-mails, appels téléphoniques ou réunions. Ces dernières
étaient quasi-hebdomadaires, afin de rester le plus possible dans la
bonne direction. Nos recontres ont eu lieu à Polytech' ou au LIRMM (Laboratoire d'Informatique Robotique Micro-électronique de Montpellier).

    \section{Gestionnaire de versions}

        \subsection{Git}
Dès le balbutiement du projet, nous avons décidé d'utiliser un gestionnaire
de version. En effet, celui-ci nous permettait d'avoir un contrôle total des modifications
tout au long du développement.

Avec le recul, nous avons eu raison puisque le nombre de lignes de codes dépasse 
les 21 000 (actuel, pas le nombre d'additions et de délétions).

Notre choix s'est naturellement porté sur Git \cite{git}. Cela s'explique 
par les aspects suivants:

    \begin{itemize}
    \item Multi-plateformes.
    \item Gestions des branches.
    \item Développement actif et forte communauté d'utilisation.
    \item Rapidité d'exécution comparé à svn, hg, etc. 
    \end{itemize}

    De plus certains d'entre nous l'avait déjà utilisé précédement, ce qui a facilité
    l'exploitation de cet outil au sein du groupe.
\newpage

        \subsection{GitHub}
    Mais surtout, le plus important pour nous reste quand même le site web GitHub \cite{github}.
    Il fait office de serveur centralisé pour git. Mais pas seulement puisqu'il 
    propose entre autre:

    \begin{itemize}
    \item L'hébergement de projets sous Git
    \item Des fonctionnalités de type réseaux sociaux, dont :
        \begin{itemize}
        \item Les flux
        \item Le suivi de personnes ou de projets
        \item Les graphes de réseau pour les dépôts
        \end{itemize}
    \item Un pastebin nommé Gist
    \item Un wiki et une page web pour chaque dépôt
    \end{itemize}

C'est un site web qualifié de professionnel puisqu'il est utilisé par de 
nombreux projets de grandes tailles tels que : Git, Perl, Facebook, Twitter,
JQuery, PHP, Python, etc.

De plus, Mr Berry, de par le LIRMM, ne pouvait nous ouvrir qu'un serveur
Subversion, donc Git couplé avec GitHub est un excellent compromis.

N.B. son adoption sous les systèmes Windows relève encore du portage
expérimental. Git a été codé par le créateur de Linux, pour versionner le code
de ce dernier, et s'utilise par conséquent en lignes de commandes.

    \section{Règles de codage}

Ci-dessous, voici les règles de codage utilisées tout au long du projet
ainsi que dans le rapport. Elles sont là pour permettre une cohérence tout au
long du développement, mais aussi et surtout, pour que la relecture et la 
compréhension soient simplifiées pour les développeurs du projet, ainsi que pour les
futures personnes amenées à travailler dessus.

Cela regroupe donc un code en anglais (commentaire compris), ne pas dépasser 80 
caractères si possible ainsin qu'une indentation de 4 espaces, pas de tabulation.
Ci-dessous, un fichier C++ d'exemple.

\begin{lstlisting}
#ifndef BADGER_FOO_HPP
#define BADGER_FOO_HPP

namespace badger
{

    class FooBar
    {
    public:
        virtual const int& doSomething() const = 0;

    private:
        int m_attribute;
    };

} // namespace badger

#endif //BADGER_FOO_HPP
\end{lstlisting}

    \section{Technologies utilisées}

        \subsection{C++}
Le langage C++ est particulièrement adapté pour notre projet puisque que celui-ci
sera amené à être proche d'un faible niveau d'abstraction sur certains points
comme la communication série/Bluetooth.

C'est aussi un langage permettant la programmation sous de multiples paradigmes
tels que:

    \begin{itemize}
    \item Programmation procédurale
    \item Programmation orienté-objet
    \item Programmation générique
    \end{itemize}

        \subsection{CMake}
CMake \cite{cmake}, pour "Cross platform Make", est un programme qui permet à partir 
de scripts uniques, de générer l'ensemble des fichiers de constructions standard 
du programme, et cela, comme son nom l'indique, de manière multi-plateforme.

En effet, CMake est capable de créer des projets pour Visual Studio ou XCode, et même 
des Makefile; et cela pour bon nombre de compilateurs (msvc, gcc, clang, etc) et 
de plateformes (Windows, MacOS, Linux, BSD, etc). Cela est possible grâce au
système de générateur de CMake.

    \begin{figure}[h]
        \begin{center}
        \includegraphics[scale=0.4]{images/CMakeFonctionnement.png} 
        \end{center}
        \caption{Fonctionnement de CMake}
        \label{Fonctionnement de CMake}
     \end{figure} 

Il est comparable au programme SCons ou aux Autotools.

C'est un projet en développement constant, puisque la dernière version date du 19
avril 2012. De plus de nombreux projets reconnus l'utilisent, comme Bullet, Ogre
ou encore KDE.

Les scripts décrivant les projets sont écrits dans les fichiers CMakeLists.txt.
Ci-dessous, un exemple minimal pour compiler un projet:

    \lstset{language=Bash}
    \begin{lstlisting} 
cmake_minimum_required(VERSION 2.6)
		
# Configuration
project(my_project)
set(EXECUTABLE_OUTPUT_PATH bin/${CMAKE_BUILD_TYPE})

file(GLOB_RECURSE source_files src/*)

add_executable(my_exe ${source_files})
    \end{lstlisting}

Le programme CMake s'utilise soit en lignes de commandes, soit avec une un logiciel
possédant une IHM\,\footnote{Interface Human Machine}.

Dans le projet, nous avons utilisé ce système de génération pour le programme 
gérant la communication avec le lecteur de carte.


        \subsection{xHTML / CSS}
Le xHTML\,\footnote{extensive HyperText Markup Langage} et le CSS\,\footnote{Cascade Style Sheet} sont
des langages de description, et non des langages de programmation.
Ces deux langages sont complémentaires, le HTML sert à représenter ``le fond'', le contenu
en lui même, alors que le CSS représente ``la forme'', c'est le design du site, permettant
une mise en page entièrement configurable via le système de balises du code HTML.

Il s'agit de deux langages simples à apprendre, car il s'agit uniquement d'imbriquer des
balises pour le HTML, et d'affecter des propriétés de styles à ces balises dans le CSS.
Toutefois, la réalisation d'un design agréable et fonctionnel n'est pas une tâche aisée,
aussi, nous avons reçu de l'aide pour le style graphique.


        \subsection{PHP}
PHP\,\footnote{PHP Hypertext Preprocessor} \cite{php} est un langage de programmation 
interprété que l'on écrit dans des fichiers de script à l'extension .php. Ce 
langage est interprété par le serveur et permet principalement la génération de 
pages web dynamiques.

Nous avons principalement utilisé le fait que PHP est devenu un langage objet dans 
sa dernière révision majeure ce qui nous a permis de construire une architecture
basée sur l'héritage pour notre framework. En plus de la programmation orientée
objet, nous avons utilisé de nombreuses fonctionnalités avancées de la librairie 
standard.


        \subsection{Javascript / AJAX}
Javascript est un langage de programmation de scripts s'exécutant chez le client.
Intégré dans la totalité des navigateurs récents, il permet la création d'applications 
web avancées (dites "riches") et donc de développer et de dynamiser les interactions
entre l'utilisateur et le serveur.

Dans notre projet, Javascript nous sert à implémenter la méthode AJAX (Asynchronous 
Javascript And XML). Elle permet l'exécution de requêtes HTTP en différé, sans 
avoir à recharger la page courante, et donc la récupération de données provenant du 
serveur. En pratique, nous l'utilisons notamment pour l'édition dynamique en ligne 
de la partie Administration, facilitant les opérations d'altération des données.

Le Javascript possède le même défaut que toutes les technologies implémentées au
niveau du navigateur, à savoir que la norme du langage n'est pas toujours respectée 
et un script codé pour un navigateur pourra s'exécuter différement sur un autre
navigateur.

Afin de faciliter le développement en Javascript, nous avons utilisé jQuery, une 
librairie développée par John Resig et fournissant une interface permettant de 
programmer en faisant abstraction de ces défauts d'implémentation. De plus, jQuery 
possède une série de fonctions complétant celles fournies par Javascript.


        \subsection{SQL / MySQL}
Le SQL est un langage permettant de communiquer avec un serveur de base de données. 
Il permet de commander l'ajout, la modification, la suppression ou encore la 
récupéraiton de données dans la base et est donc indispensable pour un application 
dynamique.

Pour ce projet, nous avons utilisé le même type de serveur que Polytech, nommé
MySQL, qui est le serveur le plus connu et le plus répandu, bien que possédant 
un certain nombre de défauts.


        \subsection{Java}
Java est un langage de programmation orienté objet. Il a la particularité de 
pouvoir fournir des programmes qui seront assez facilement portables sur plusieurs 
plateformes. Il présente une certaine ressemblance au C++, avec quelques nuances 
comme l'absence de la gestion de la mémoire via les pointeurs ou encore l'accent 
sur l'aspect orienté objet. Nous avons été amenés à l'utiliser dans le cadre de 
la modification de PDFBox. 


       \subsection{UML}
UML ne correspond pas à un langage de progammation, mais plutôt à un langage 
graphique de modélisation. Il a été créé dans le but d'harmoniser les
différentes méthodes de modélisation. Aujourd'hui il s'impose comme un standard. 
Nous l'avons choisi principalement pour 2 raisons car il correspond à l'un des 
enseignements suivi cette année et parce que c'est un standard.



    \section{Juridique}
Comme le projet est voué à être ré-utilisé par la suite, tout le code que nous 
avons écrit se trouve sous licence BSD, pour "Berkeley software distribution
license". Le texte complet de la licence est contenu en annexe.

Dans l'ensemble de nos répertoires Git, la licence se trouve à la racine, dans les
fichiers COPYING. Dans les fichiers compilés (.cpp, .hpp, etc), on la retrouve
également en en-tête, contrairement aux fichiers lus très fréquemment, comme les
scripts PHP et Javascript.

Cela lui confère entre autres d'être libre et elle permet de réutiliser tout ou 
une partie du logiciel sans restriction, qu'il soit intégré dans un logiciel libre 
ou propriétaire.

