\chapter{Introduction}
Le projet "Semaine du numérique" que nous devons réaliser est l’organisation
d'un nouvel enseignement axé autour de l'informatique pour Polytech' Montpellier.

En effet, la CTI (Comission des Titres d'Ingénieurs) considère que les 
connaissances des élèves mériteraient d'être approfondie dans le domaine de l'informatique.

Cet enseignement sera comparable à celui des Langues vivantes, ainsi que des
SHEJS (Science Humaines, Economiques, Juridiques et Sociales), et sera suivi
par les élèves de tous les départements - excepté IG, au cours de leur cycle 
ingénieur, au cours de l'année 4 ou 5.

Ce choix sera fait pas les directeurs du département eux-mêmes. L'unité 
d'enseignement aura lieu au cours d'une semaine banalisée où les élèves 
devront assisté à un certain nombre de conférences, traitant des thématiques 
suivantes:

\begin{itemize}
\item Système d’information : ERP
\item Maîtrise d’ouvrage
\item Sécurité
\item Collecticiels
\item Architecture système
\item Valeurs juridiques (dépendant de SHEJS)
\end{itemize}

A la fin de cette semaine, les élèves seront évalués sur un QCM personnalisé 
en fonction des conférences suivies. De plus l'assiduité aux conférences influera
sur la note finale.

Notre projet se scinde alors en deux parties distinctes:
\begin{itemize}
\item Lecteur de carte pour contrôler la présence.
\item Site web pour organiser insciptions et QCM.
\end{itemize}
